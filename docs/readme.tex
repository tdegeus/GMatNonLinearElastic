\documentclass{goose-article}

\hypersetup{pdfauthor={T.W.J. de Geus}}

\title{Non-linear elasticity}

\author[1]{Tom de Geus}

\begin{document}

\maketitle

\begin{abstract}
This constitutive model encompasses a non-linear, but history independent, relation between the Cauchy stress, $\bm{\sigma}$, and the linear strain tensor, $\bm{\varepsilon}$, i.e.:
\begin{equation*}
  \bm{\sigma} = f \left( \bm{\varepsilon} \right)
\end{equation*}
The model is implemented in 3-D, hence it can directly be used for either 3-D or 2-D plane strain problems.
\end{abstract}

\section{Constitutive model}

The following strain-energy is defined:
%
\begin{equation}
  U ( \bm{\varepsilon} )
  = \frac{9}{2} K \varepsilon_\mathrm{m}
  + \frac{ \sigma_0 \, \varepsilon_0 }{ n+1 }
    \left( \frac{\varepsilon_\mathrm{eq}}{\mathrm{\varepsilon_0}} \right)^{n+1}
\end{equation}
%
where $K$ is the bulk modulus, $\varepsilon_0$ and $\sigma_0$ are a reference strain and stress respectively, and $n$ is an exponent that sets the degree of non-linearity. Finally $\varepsilon_\mathrm{m}$ and $\varepsilon_\mathrm{eq}$ are the hydrostatic and equivalent strains (see Appendix~\ref{sec:ap:strain}).

This leads to the following stress--strain relation:
%
\begin{equation}
\label{eq:stress}
  \bm{\sigma}
  = \frac{\partial U}{\partial \bm{\varepsilon}}
  = 3 K \varepsilon_\mathrm{m} \, \bm{I}
  + \frac{2}{3} \frac{\sigma_0}{\varepsilon_0^n} \,
    \varepsilon_\mathrm{eq}^{n-1} \, \bm{\varepsilon}_\mathrm{d}
\end{equation}
%
see Appendix~\ref{sec:ap:nomenclature} for nomenclature.

\section{Consistent tangent}

The consistent tangent maps a variation in strain, $\delta \bm{\varepsilon}$, to a variation in stress, $\delta \bm{\sigma}$, as follows
%
\begin{equation}
  \delta \bm{\sigma} = \mathbb{C} : \delta \bm{\varepsilon}
\end{equation}
%
The tangent, $\mathbb{C}$, thus corresponds to the derivative of Eq.~\eqref{eq:stress} w.r.t.\ strain. For this, the chain rule is employed:
%
\begin{equation}
  \mathbb{C}
  = \frac{\partial}{\partial \bm{\varepsilon}}
    \bigg[\;
      3 K \varepsilon_\mathrm{m} \, \bm{I}
    \;\bigg]
  + \frac{\partial}{\partial \bm{\varepsilon}_\mathrm{d}}
    \bigg[\;
      \frac{2}{3} \frac{\sigma_0}{\varepsilon_0^n} \,
      \varepsilon_\mathrm{eq}^{n-1} \, \bm{\varepsilon}_\mathrm{d}
    \;\bigg]
  : \frac{\partial \bm{\varepsilon}_\mathrm{d}}{\partial \bm{\varepsilon}}
\end{equation}
%
Where:
\begin{itemize}
%
\item the derivative of the volumetric part reads
%
\begin{equation}
  \frac{\partial}{\partial \bm{\varepsilon}}
  \bigg[\;
    3 K \varepsilon_\mathrm{m} \, \bm{I}
  \;\bigg]
  = K \bm{I} \otimes \bm{I}
\end{equation}
%
\item the chain rule for the deviatoric part reads
%
\begin{align}
  \frac{\partial}{\partial \bm{\varepsilon}_\mathrm{d}}
  \bigg[\;
    \varepsilon_\mathrm{eq}^{n-1} \, \bm{\varepsilon}_\mathrm{d}
  \;\bigg]
  &=
  \frac{
    \partial \big[ \varepsilon_\mathrm{eq}^{n-1} \big]
  }{
    \partial \bm{\varepsilon}_\mathrm{d}
  } \otimes \bm{\varepsilon}_\mathrm{d}
  + \varepsilon_\mathrm{eq}^{n-1}
  \frac{
    \partial \bm{\varepsilon}_\mathrm{d}
  }{
    \partial \bm{\varepsilon}_\mathrm{d}
  }
  \\
  &=
  \tfrac{2}{3} (n-1) \, \varepsilon_\mathrm{eq}^{n-3} \,
  \bm{\varepsilon}_\mathrm{d} \otimes \bm{\varepsilon}_\mathrm{d}
  + \varepsilon_\mathrm{eq}^{n-1} \, \mathbb{I}
\end{align}
%
\item and it has been used that
%
\begin{align}
  \frac{\partial}{\partial \bm{\varepsilon}_\mathrm{d}}
  \bigg[\;
    \varepsilon_\mathrm{eq}^{n-1}
  \;\bigg]
  &= (n-1)\, \varepsilon_\mathrm{eq}^{n-2} \,
  \frac{2}{3} \frac{\bm{\varepsilon}_\mathrm{d}}{\varepsilon_\mathrm{eq}}
  \\
  &= \tfrac{2}{3} (n-1) \,
    \varepsilon_\mathrm{eq}^{n-3} \, \bm{\varepsilon}_\mathrm{d}
\end{align}
%
\end{itemize}
%

Combining the above yields:
\begin{align}
\mathbb{C}
&= K \bm{I} \otimes \bm{I} +
\frac{2}{3} \frac{\sigma_0}{\varepsilon_0^n}
\bigg(
  \tfrac{2}{3} (n-1) \, \varepsilon_\mathrm{eq}^{n-3}
  \bm{\varepsilon}_\mathrm{d} \otimes \bm{\varepsilon}_\mathrm{d}
  + \varepsilon_\mathrm{eq}^{n-1} \mathbb{I}
\bigg) : \mathbb{I}_\mathrm{d} \\
&= K \bm{I} \otimes \bm{I} +
\frac{2}{3} \frac{\sigma_0}{\varepsilon_0^n}
\bigg(
  \tfrac{2}{3} (n-1) \, \varepsilon_\mathrm{eq}^{n-3}
  \bm{\varepsilon}_\mathrm{d} \otimes \bm{\varepsilon}_\mathrm{d}
  + \varepsilon_\mathrm{eq}^{n-1} \, \mathbb{I}_\mathrm{d}
\bigg)
\end{align}

\section{Consistency check}

To check if the derived tangent $\mathbb{C}$ a \emph{consistency check} can be performed. A (random) perturbation $\delta \bm{\varepsilon}$ is applied. The residual is compared to that predicted by the tangent. For the general case of linearisation, the following holds:
%
\begin{equation}
  \bm{\sigma}\big( \bm{\varepsilon}_\star + \delta \bm{\varepsilon} \big) =
  \bm{\sigma}\big( \bm{\varepsilon}_\star \big) +
  \mathbb{C} \big( \bm{\varepsilon}_\star \big) : \delta \bm{\varepsilon} +
  \mathcal{O}(\delta \bm{\varepsilon}^2)
\end{equation}
%
or
%
\begin{equation}
  \underbrace{
    \bm{\sigma}\big( \bm{\varepsilon}_\star + \delta \bm{\varepsilon} \big) -
    \bm{\sigma}\big( \bm{\varepsilon}_\star \big)
  }_{
    \displaystyle \delta \bm{\sigma}
  } -
  \mathbb{C} \big( \bm{\varepsilon}_\star \big) : \delta \bm{\varepsilon} =
  \mathcal{O}(\delta \bm{\varepsilon}^2)
\end{equation}
%
This allows the introduction of a relative error
%
\begin{equation}
  \eta =
  \Big|\Big|
    \delta \bm{\sigma} -
    \mathbb{C}(\bm{\varepsilon}_\star) : \delta \bm{\varepsilon}
  \Big|\Big|
  /
  \Big|\Big| \delta \bm{\sigma} \Big|\Big|
\end{equation}
%
This \emph{truncation error} thus scales as $\eta \sim || \delta \bm{\varepsilon} ||^2$ as depicted in Figure~\ref{fig:consistency:expected}. As soon as the error becomes sufficiently small the numerical \emph{rounding error} becomes more dominant, the scaling thereof is also included in Figure~\ref{fig:consistency:expected}.

\begin{figure}[htp]
  \centering
  \includegraphics[width=.5\textwidth]{figures/consistency}
  \caption{Expected behaviour of the consistency check, see \citet[p.~9]{Heath2002}.}
  \label{fig:consistency:expected}
\end{figure}

The measurement of $\eta$ and a function of $|| \delta \bm{\varepsilon} ||$, as depicted in Fig.~\ref{fig:consistency}, indeed matches the prediction in Fig.~\ref{fig:consistency:expected}.

\begin{figure}[htp]
  \centering
  \includegraphics[width=.5\textwidth]{examples/consistency}
  \caption{Measured consistency check, cf.\ Fig.~\ref{fig:consistency:expected}.}
  \label{fig:consistency}
\end{figure}

\bibliography{library}

\appendix
\vfill\newpage

\section{Nomenclature}
\label{sec:ap:nomenclature}

\paragraph{Tensor products}
\vspace*{.5eM}

\begin{itemize}
%
\item Dyadic tensor product
\begin{align}
  \mathbb{C} &= \bm{A} \otimes \bm{B} \\
  C_{ijkl}   &= A_{ij} \,      B_{kl}
\end{align}
%
\item Double tensor contraction
\begin{align}
  C &= \bm{A} : \bm{B} \\
    &= A_{ij} \, B_{ji}
\end{align}
%
\end{itemize}

\paragraph{Tensor decomposition}
\vspace*{.5eM}

\begin{itemize}
%
\item Deviatoric part $\bm{A}_\mathrm{d}$ of an arbitrary tensor $\bm{A}$:
\begin{equation}
  \mathrm{tr}\left( \bm{A}_\mathrm{d} \right) \equiv 0
\end{equation}
and thus
\begin{equation}
  \bm{A}_\mathrm{d} = \bm{A} - \tfrac{1}{3} \mathrm{tr}\left( \bm{A} \right)
\end{equation}
%
\end{itemize}

\paragraph{Fourth order unit tensors}
\vspace*{.5eM}

\begin{itemize}
%
\item Unit tensor:
\begin{equation}
  \bm{A} \equiv \mathbb{I} : \bm{A}
\end{equation}
and thus
\begin{equation}
  \mathbb{I} = \delta_{il} \delta{jk}
\end{equation}
%
\item Right-transposition tensor:
\begin{equation}
  \bm{A}^T \equiv \mathbb{I}^{RT} : \bm{A} = \bm{A} : \mathbb{I}^{RT}
\end{equation}
and thus
\begin{equation}
  \mathbb{I}^{RT} = \delta_{ik} \delta_{jl}
\end{equation}
%
\item Symmetrisation tensor:
\begin{equation}
  \mathrm{sym} \left( \bm{A} \right) \equiv \mathbb{I}_\mathrm{s} : \bm{A}
\end{equation}
whereby
\begin{equation}
  \mathbb{I}_\mathrm{s} = \tfrac{1}{2} \left( \mathbb{I} + \mathbb{I}^{RT} \right)
\end{equation}
This follows from the following derivation:
\begin{align}
  \mathrm{sym} \left( \bm{A} \right) &= \tfrac{1}{2} \left( \bm{A} + \bm{A}^T \right)
  \\
  &= \tfrac{1}{2} \left( \mathbb{I} : \bm{A} + \mathbb{I}^{RT} : \bm{A} \right)
  \\
  &= \tfrac{1}{2} \left( \mathbb{I} + \mathbb{I}^{RT} \right) : \bm{A}
  \\
  &= \mathbb{I}_\mathrm{s} : \bm{A}
\end{align}
%
\item Deviatoric and symmetric projection tensor
\begin{equation}
  \mathrm{dev} \left( \mathrm{sym} \left( \bm{A} \right) \right) \equiv \mathbb{I}_\mathrm{d} : \bm{A}
\end{equation}
from which it follows that:
\begin{equation}
  \mathbb{I}_\mathrm{d}
  = \mathbb{I}_\mathrm{s} - \tfrac{1}{3} \bm{I} \otimes \bm{I}
\end{equation}
%
\end{itemize}

\section{Strain measures}
\label{sec:ap:strain}

\begin{itemize}
%
\item Mean strain
\begin{equation}
  \varepsilon_\mathrm{m}
  = \tfrac{1}{3} \, \mathrm{tr} ( \bm{\varepsilon} )
  = \tfrac{1}{3} \, \bm{\varepsilon} : \bm{I}
\end{equation}
%
\item Strain deviator
\begin{equation}
  \bm{\varepsilon}_\mathrm{d}
  = \bm{\varepsilon} - \varepsilon_\mathrm{m} \, \bm{I}
  = \mathbb{I}_\mathrm{d} : \bm{\varepsilon}
\end{equation}
%
\item Equivalent strain
\begin{equation}
  \varepsilon_\mathrm{eq}
  = \; \sqrt{
    \tfrac{2}{3} \, \bm{\varepsilon}_\mathrm{d} : \bm{\varepsilon}_\mathrm{d}
  }
\end{equation}
%
\end{itemize}

\section{Variations}
\label{sec:ap:variations}

\begin{itemize}
%
\item Strain deviator
\begin{equation}
  \delta \bm{\varepsilon}_\mathrm{d}
  = \left( \mathbb{I}_\mathrm{s} - \tfrac{1}{3} \bm{I} \otimes \bm{I} \right) :
    \delta \bm{\varepsilon}
  = \mathbb{I}_\mathrm{d} : \delta \bm{\varepsilon}
\end{equation}
%
\item Mean equivalent strain
\begin{equation}
  \delta \varepsilon_\mathrm{m}
  = \tfrac{1}{3} \bm{I} : \delta \bm{\varepsilon}
\end{equation}
%
\item Von Mises equivalent strain
\begin{align}
  \delta \varepsilon_\mathrm{eq}
  &= \frac{1}{3} \frac{1}{\varepsilon_\mathrm{eq}}
     \left( \bm{\varepsilon}_\mathrm{d} : \delta \bm{\varepsilon}_\mathrm{d} +
     \delta \bm{\varepsilon}_\mathrm{d} : \bm{\varepsilon}_\mathrm{d} \right) \\
  &= \frac{2}{3} \frac{1}{\varepsilon_\mathrm{eq}}
     \left( \bm{\varepsilon}_\mathrm{d} : \delta \bm{\varepsilon}_\mathrm{d} \right) \\
  &= \frac{2}{3} \frac{\bm{\varepsilon}_\mathrm{d}}{\varepsilon_\mathrm{eq}} :
     \delta \bm{\varepsilon}_\mathrm{d}
\end{align}
%
\end{itemize}

\end{document}
